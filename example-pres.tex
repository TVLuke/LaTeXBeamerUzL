\documentclass[german,10pt,xcolor=colortbl,compress
%,draft
]{beamer}
\usepackage{xunicode}
\usepackage[T1]{fontenc}
\usepackage{calc}
\usepackage[ngerman]{babel} % Neue Rechtschreibung
\usepackage{amsmath,amsthm,amssymb,euscript} % AMS-LaTeX  
\usepackage{enumerate,graphicx}

% Load Them
\usetheme[footline=true, slogan=false, navigation=true,ITMtheme]{UzL}
%
\setbeamertemplate{navigation symbols}{}
\title{Beispielpräsentation}
\subtitle{Eines Vortrages im neuen Design mit \LaTeX}
\date[]{5. Februar 2011\\[1ex] Workshop on Corporate Design Stuff}
\author[R. Bergmann]{Ronny Bergmann}
% Clear Logo 1 to make the head smaller
%\clearlogo{1}
\institute[Universität zu Lübeck]{Institut für Mathematik\\Universität zu Lübeck}
\begin{document}
	{%in brackets to keep changes local
	\setbeamercolor{background canvas}{bg=black}
	\setbeamercolor{text-color}{fg=white}
	\begin{frame}[plain]{}{}%Ohne Titel und Untertitel damits ganz leer und weiß ist
		\color{white}\small .
	\end{frame}
	}
	\maketitle	
	\section{Einleitung}
	\begin{frame}{}
		\tableofcontents
	\end{frame}
	\subsection{A}
	\begin{frame}{Einleitung - Motivation}
	ABC
	\begin{lemma}[Ein Beispiellemma]
		Ist das hier und es gilt.
	\end{lemma}
	\begin{example}
		Ein Beispiel, wie dieses hier
	\end{example}
	\alert{ACHTUNG!}
		Etwas Hervorgehobenes. Die Farben sind bisher alle dem Handbuch zum Corporate Design entnommen.
	\end{frame}
	\subsection{B}
	\begin{frame}{Ein Beweis}{Mit Untertitel}
			\begin{proof}
				Weil.
			\end{proof}
	\end{frame}
	\begin{frame}[plain]{}{}%Ohne Titel und Untertitel damits ganz leer und weiß ist
		Ich bin so ein leerer Frame
	\end{frame}
	
	\begin{frame}{TODO}
		\begin{itemize}
			\item Literatur?
		\end{itemize}
	\end{frame}
\end{document}
